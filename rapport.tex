\documentclass[a4paper, openany, 11pt]{article}

%%% PACKAGES %%%
\usepackage[left=2cm, right=2cm, top=2.5cm, bottom=2.5cm]{geometry}
\usepackage[french]{babel}
\usepackage[utf8]{inputenc}
\usepackage{amsmath}
\usepackage[cal=boondox]{mathalfa}
\usepackage{bm}
\usepackage{amsfonts}
\usepackage{amssymb}
\usepackage{textcomp}
\usepackage{nccmath}
\usepackage{algorithmicx}
\usepackage{algpseudocode}
%\renewcommand{\theequation}{\thesubsection\arabic{equation}}
\usepackage{caption}
\usepackage{subcaption}
\usepackage{graphicx}
%\makeatletter
%\graphicspath{{./Images/}}
%\def\input@path{{./Images/}}
%%\def\input@path{{Images_light/}}
%\makeatother
\usepackage{tikz}
\usepackage{hyperref}
\hypersetup{
    colorlinks,
    linkcolor={red!50!black},
    citecolor={blue!50!black},
    urlcolor={blue!80!black}
}
  
\usepackage{multirow}
\usepackage{multicol}

\newcommand{\vr}{\bm{r}}
\newcommand{\kb}{\mathcal{k}_B}

\title{Simulation du modèle XY par méthode Monté-Carlo \medskip \\ 
       \Large Rapport de projet pour le cours de Computationnal Statistical Physics }
\author{Aurélien Valade}
\date{}

\begin{document}
\maketitle

{\centering \rule{\textwidth}{.5pt} }

\section*{Introduction}

La physique statistique est l'étude des comportements systèmes macroscopiques composés de systèmes
microscopiques dont les propriétés sont connues. On va donc systématiquement s'intéresser à un grand
nombre de particules en interactions, ce qui est complexe à représenter mathématiquement sans faire
approximation de continuité, de champ moyen, etc. Ainsi, les comportements de nombreux modèles de
physique statistique ne sont pas discriptibles analytiquement pour des cas pourtant pertinents.  Si
une résolution analytique n'est pas possible, il faut travailler avec des valeurs numériques, comme
cela a par exemple était fait pendant la seconde guerre mondiale pour les calculs portant sur la
bombe atomique, ou encore en astrophysique pour des calculs de trajectoires.  Cependant la quantité
de calcul nécessaire pour résoudre des systèmes réalistes est tout bonnement inhumaine.

L'informatique nous offre une capacité de calcul immensément plus grande que celle des calculatrices
humaines, puisqu'on passe d'une dizaine d'opérations par minute à des milliards par seconde. Mais ce
n'est toujours pas suffisant pour attaquer le problème de front, par exemple, calculer directement
la fonction de partition du système reste souvent hors de portée. Il nous faut donc trouver des
méthodes plus puissantes pour résoudre le système. Cette formalisation à cependant un prix puisque
de telles méthodes sont alors plus spécifiques : une méthode de résolution est adaptée à une
certaine question, mais ne permet pas toujours de bien répondre à d'autres. On pourra ainsi
rapidement calculer des propriétés d'un système à l'équilibre sans pouvoir décrire la dynamique
physique de la relaxation. Réciproquement, une autre méthode peut permettre de décrire la dynamique
du système avec précision, mais demande une très grande quantité de calcul pour atteindre l'état
d'équilibre. 

Nous allons ici nous pencher sur l'étude du modèle XY, qui est un modèle d'interaction sur réseau se
rapprochant du modèle d'Ising, à la différence que les états pris par les spins sont continus entre
-1 et +1 et non restreints aux valeurs $\{-1, +1\}$. Il serait cependant faux de dire que le modèle
XY est une généralisation du modèle d'Ising puisque pour un système équivalent, aucun choix de
paramètres du modèle XY ne permet de reproduire des comportements semblables à ceux conjecturés par
modèle d'Ising. Le modèle XY est donc adapté à d'autres systèmes que le modèle d'Ising : il permet
notamment de décrire des systèmes dont le paramètre d'ordre ne correspond pas à une brisure de
symétrie spatiale lors de la transition de phase, mais plutôt à une brisure de la symétrie interne,
comme c'est par exemple le cas pour  l'Helium II supercritique ou encore les cristaux liquides
hexatiques\footnote{La phase hexatique est une phase située entre les phases solide isotropique et
liquide isotropique dans les systèmes de particules à deux dimensions (cristal liquide
bidimensionnel). Il est caractérisés par deux paramètres d'ordre : un ordre moléculaire de
positionnement à courte portée et un ordre orientationnel à longue portée
\cite{wiki:phase_hexatique}.}.

Nous allons dans un premier temps présenter le problème dans sa formulation physique.
 
\section{Modèle physique}

\subsection{Description du modèle}

Soit un ensemble de particules arrangées sur une grille carrée de taille $N\times N$. On note la
position $\vr{ij}$ pour la particule de la ligne $0 \le i < N$ et la colonne $0 \le j < N$. On
utilise aussi l'indiciation des particules $0 \le I < N\times N$ où la position spiatiale n'apparaît
pas explicitement bien qu'on puisse définir $I = i\times N +j$.

Ces particules ont une propriété appelée spin que l'on peut lier à l'angle entre la normale au plan
et un axe intrinsèque à la particule. On note cet angle $\theta_{ij} \in [0, \pi]$. On peut alors
définir le spin de chaque particule $\sigma_{ij} = \cos(\theta_{ij})$. On considère que ces
particules interagissent entre elles à travers leur spin, et qu'il existe un champ exterieur avec
lequel ces spin tendent à s'aligner. Le Hamilonien décrivant l'énergie d'un tel système s'écrit
alors : 
\begin{equation}
    \begin{aligned}  
        H &= - k \underbrace{\sum_{I, I'} J_{II'} \cos(\theta_I-\theta_{I'})}_{\text{interactions}} 
        - h \underbrace{\sum_I \cos(\theta_I) }_{\text{champ exterieur}} \\
          &= -k Q - h M \\
    \end{aligned}
\end{equation}
avec
\begin{equation}
    Q = \sum_{I, I'} J_{II'} \cos(\theta_I-\theta_{I'})~, \qquad M = \sum_I \cos(\theta_I) 
\end{equation}
nommées respectivement le couplage et la magnétisation par analogie aux systèmes magnétiques.

La forme de la matrice $J$ est centrale au problème. Le modèle XY préconise de prendre une fonction
de la distance entre les deux particules : $J_{II'} = (\vr_I - \vr_{I'})^{-\alpha}$. On peut ensuite
choisir de n'appliquer cette définition qu'aux $n$ plus proches voisins et de négliger les
interactions à longue distance : 
\begin{equation}
    J_{II'} = 
    \begin{cases}
        (\vr_I - \vr_{I'})^{-\alpha} & \text{si~} |i-i'|<n \text{~et~} |j-j'|<n, \\
        0 & \text{sinon.}
    \end{cases}
\end{equation}

Les paramètres $h$ et $k$ vont nous permettre permettent d'adapter ce système à différents cas
concrets. La valeur de $h$ représente l'intensité du champ externe auquel les spins tendent à
s'aligner, et $k$ modélise la force du couplage entre les spins. Pour un modèle magnétique, $k>0$
signifie que le matériau est ferromagnétique : les spins cherchent à pointer dans la même direction
et $k>0$ représente un matériau paramagnétique : les spins cherchent à pointer dans une direction
opposée à celle de leurs voisins. Ces quantités ont été préalablement adimensionnées, et dérivent de
termes de la forme $h=h'/(\kb T)$, $k = k'/(\kb T)$ où $\kb$ est la constante de Boltzmann et $T$ la
température. C'est donc à travers ces paramètres qu'on peut jouer sur la température pour observer
les transitions de phase, etc. 

\subsection{Comportement prédit}

A haute température les paramètres $h$ et $k$ du système sont très faibles, le Hamiltonien résultant
est donc quasi-nul et n'a pas de minimum significatif. Il n'y a pas d'état qui puisse réellement
minimiser l'énergie, tous sont donc équiprobaux, il en résulte que la magnétisation ainsi que le
couplage sont nuls en moyenne.

À basse température, le comportement du modèle XY est différent du modèle d'Ising. Tout d'abord, lors
de la transition de phase, on n'observe pas de brisure de symétrie spatiale. A champs magnétique
externe nul $h=0$, lors d'une trempe (\emph{i.e.} lors qu'on abaisse progressivement la température du
système), la magnétisation reste nulle. En revanche, en présence d'un champ externe, la symmétrie du
système étant déjà brisée, et les spins s'alignent effectivement avec ce champ. 

On observe à la transition de phase un changement globale de comportement de la fonction de corrélation
\begin{equation}
    C_{II'} = \big|\big< \sigma_I \sigma_{I'} \big>\big|.
\end{equation}

En effet, à haute température, la fonction de corrélation décroît exponentiellement :
\begin{equation}
    C_{II'}^{HT} \propto e^{\Gamma(\beta) |\vr_I - \vr_{I'}|}
\end{equation}
alors qu'à basse température, on s'attend à trouver une fonction qui décroît moins vite que 
\begin{equation}
    C_{II'}^{BT} \propto \frac{\Gamma(\beta)}{1+|\vr_I - \vr_{I'}|}.
\end{equation}



\section{Implémentation}

Le but de ce travail est de fournir un programme puissant, flexible, et aisément utilisable qui
permette de simuler et d'analyser un modèle XY classique par une méthode Monté-Carlo. Il doit
notamment utiliser un système de production, de stockage et d'analyse des données cohérent qui
permette de reproduire chaque experience. Il faut donc notamment ne pas écraser les données au fur
et à mesure qu'elles sont produite, mais il faut en plus garder des traces des paramètres associés à
chaque experience produite, ainsi que des conventions de nomages compréhensibles et cohérentes.

\subsection{Structure du code et des scipts}

Le code produit se décompose en deux parties distinctes :
\begin{itemize}
    \item le code principal d'environs 500 lignes, écrit en C  : il produit les données pour une expérience
        et les stocke dans un dossier ;
    \item et un code de scirpting écrit en Python d'environs 600 lignes : celui ci permet d'appeler le code C avec
        les bons paramètres puis d'exploiter les données qui en résultent.
\end{itemize}

\subsection{Fonctionnement du code principal}

\subsubsection{Entrés et sorties}

Chaque expérience est lancée avec la connaissance des paramêtres suivant :
\begin{itemize}
    \item $N$ la taille de grille ;
    \item $n$ le nombre de plus proches voisins à considérer pour les intéractions ;
    \item $a$ l'exposant de la fonction de décroissance pour les intéractions ;
    \item $h$ la force du champ exterieur ;
    \item $k$ la force de couplage entre spins ;
    \item le type d'initialisation :
        \begin{itemize}
            \item \texttt{UP} : $\theta_{I}=0$ pour tout $I$ ;
            \item \texttt{DOWN} : $\theta_{I}=\pi$ pour tout $I$ ;
            \item \texttt{CHECKERBOARD} : $\theta_{ij}=\pi |\mathrm{mod}(i,2)-\mathrm{mod}(j, 2)|$ pour tout $i, j$ ;
            \item \texttt{RANDOM} : $\theta_{I}= U([0, \pi])$ pour tout $I$ avec $U$ la loi
                uniforme. 
        \end{itemize}
    \item le type de Monte-Carlo : \texttt{METROPOLIS}  ou \texttt{GLAUBER}.
    \item Ainsi que le nombre de pas, la fréquence de sauvegarde des différentes données et le
        répertoire où les stocker.
\end{itemize}
En sortie, le programme fournit les données suivantes :
\begin{itemize}
    \item Un fichier où sont stockées les valeurs des quantités $M$, $Q$ et $E$ toutes les
        \texttt{n\_sca} itérations ; 
    \item Un fichier toutes les \texttt{n\_lat} itérations, représentant l'ensemble des angles de la
        grille ; 
    \item Un fichier toutes les \texttt{n\_lat} itérations, représentant l'ensemble des interactions
        $Q_I$ de la grille.
\end{itemize}

\subsubsection{Definition des quantités numériques}

Les quantités physiques introduites prédemment sont définies numériquement comme suit :
\paragraph{Magnétisation.}
La définition de la magnétisation est similaire à sa définition physique : 
\[
M = \sum_I \cos(\theta_I).
\] On a donc $M \in [-N\times N, N \times N]$.
\paragraph{Couplage.}
On introduit la matrice de couplage normalisée telle que
\begin{equation*}
    \hat{J}_{II'} = \frac{J_{II'}}{\sum_{KK'} J_{KK'} }.
\end{equation*}
Cette définition permet de garder un couplage indépendant du nombre de voisins considérés.  En
effet, en absence de normalisation, pour une configuration donnée, l'énergie de couplage de chaque
particule grandit fortement avec le nombre de plus proches voisins considérés, ce qui peut s'avérer
gênant pour comparer des simulations à $n$ différent.  On introduit ensuite le couplage local 
\begin{equation*}
    Q_I = \sum_{I'} \hat{J}_{II'} ( \cos \theta_I \cos \theta_{I'}  -
    \sin \theta_I \sin \theta_{I'} )
\end{equation*}
tel que $Q = \sum_I Q_I$, et dont les valeurs qui sont stockées sur la grille des interactions
introduite plus haut. Développer le cosinus en une différence de produits de cosinus/sinus permet de
stocker les valeurs de cosinus et de sinus pour chaque angle et de calculer un minimum de fonctions
trigonométriques à chaque mise à jour. Avec la normalisation de la matrice de couplage, on a aussi
$Q \in [-N\times N, N \times N]$.
\paragraph{Énergie.} On définie l'énergie du système 
\begin{equation*}
    E = -k Q - h M.
\end{equation*}

\paragraph{Quantitées réduites.} Pour pouvoir comparer des experiences dont la taille de la grille
diffère, on utilise les valeurs réduites :
\begin{equation*}
    m = M / N^2, \quad q = Q / N^2, \quad e = E / N^2.
\end{equation*}
Ces valeurs sont particulièrement stables puisqu'on a $m,q \in [-1, 1]$ et $e \in [-|h|-|k|,
|h|+|k|]$ pour toute expérience, quelque soient ses paramètres. 


\subsubsection{Algorithmes utilisés}

Après initialisation des différentes parties du code, on procède à l'exploration de l'espace des
phases par méthode de Monté-Carlo comme vue en cours. À chaque itération, une cellule de la grille
est choisie au hasard à laquelle on assigne un angle aléatoire. On recalcule l'énergie de la
nouvelle configuration. On propose ensuite deux méthodes
\begin{itemize}
    \item  \texttt{METROPOLIS} : la modification est acceptée avec une probabilité de $\min(1, \exp(-\Delta
        E))$, sinon elle est rejetée ;
    \item \texttt{GLAUBER} : la modification est acceptée avec une probabilité de $1/(1+\exp(\Delta
        E)$, sinon elle est rejetée.
\end{itemize}

La procédure est décrite en pseudo-code dans la figure \ref{fig:update}.
\begin{figure}[h]
    \fbox{
    \begin{minipage}[]{\textwidth}
        \begin{algorithmic}
            \State $i \gets  $ int\_uniform\_random(0, $N-1$)
            \Comment{ligne à modifier}
            \State $j \gets  $ int\_uniform\_random(0, $N-1$)
            \Comment{colonne à modifier}
            \State $a \gets  $ uniform\_random(0, $\pi$)
            \Comment{nouvel angle}
            \State $p \gets  $ uniform\_random(0, 1)
            \Comment{probabilité d'acceptation} \medskip
            \State $b \gets  $ get\_lattice\_value(lattice, $i$, $j$)
            \Comment{sauvegarde de l'angle}
            \State $E \gets  $ get\_energy(lattice)
            \Comment{sauvegarde de l'énergie} \medskip
            \State set\_lattice\_value(lattice, $i$, $j$, $a$)
            \Comment{assignation du nouvel angle}
            \State compute\_energy(lattice)
            \Comment{calcul de l'énergie dans la nouvelle configuration}
            \State $\Delta E \gets $ get\_energy(lattice) $ - E$ \medskip
            \If {monte\_carlo=\texttt{METROPOLIS}}
                \If {$\exp(-\Delta E) < p$} 
                \Comment{si le pas est rejeté}
                    \State set\_lattice\_value(lattice, $i$, $j$, $b$)
                    \Comment{réassignation de l'ancienne valeur d'angle}
                    \State set\_energy(lattice, $E$)
                    \Comment{réassignation de l'ancienne valeur d'énergie}
                \EndIf
            \ElsIf {monte\_carlo=\texttt{GLAUBER}}
                \If {$1/(1+\exp(\Delta E)) < p$} 
                \Comment{si le pas est rejeté}
                    \State set\_lattice\_value(lattice, $i$, $j$, $b$)
                    \Comment{réassignation de l'ancienne valeur d'angle}
                    \State set\_energy(lattice, $E$)
                    \Comment{réassignation de l'ancienne valeur d'énergie}
                \EndIf
            \EndIf
        \end{algorithmic}

    \end{minipage}
    }
    \caption{Procédure de mise à jour de la grille en pseudo code.}
    \label{fig:update}
\end{figure}

Une fois le spin d'une particule mis à jour, il faut mettre à jour toutes les quantitées, a savoir
$M$, $Q$ et $E$. Il s'agit de la partie du code qui demande le plus de calcul. Plus précisément,
c'est le calcul du couplage qui est le plus lourd, et l'est d'autant plus que l'on considère les
actions à longue distance.

Une approche naive serait de changer recalculer la magnétisation et du couplage après chaque
modification sur toute la grille ; une telle approche serait de complexité $\mathcal{O}(n^2\times
N^2)$.  Une meilleur algorithme, celui-ci en $\mathcal{O}(n^2)$ est décrit en figure \ref{fig:calc}.
L'idée consiste à ne recalculer que la partie de grille affectée par le changement et de reporter
directement les modifications sur les scalaires $M$ et $Q$. Le poids final de chaque opération est
reporté en figure \ref{fig:perfdata}.

\begin{figure}
    \fbox{
    \begin{minipage}[]{\textwidth}
        \begin{algorithmic}
            \Require $a, b$ la nouvelle/ancienne valeur de l'angle modifié
            \Require $i, j$ la position de l'angle modifié
            \Require $M, Q, E$ la magnétisation, le couplage et l'énergie \medskip
            \State $M \gets M - \cos(b) + \cos(a)$ \medskip
            \Comment{Mise à jour de la magnétisation}
            \State $p \gets 0,~q\gets 0$
            \Comment{Ancienne/nouvelle valeur du couplage local}
            \For { $i', j'$ in the neighborhood of $i, j$ } 
                \Comment{ Boucle sur les particules en interaction }
                \State $p \gets p \,+ $ get\_interaction\_value($i'$, $j'$)
                \Comment{ Ancienne valeur }
                \State compute\_interaction\_one\_particule($i'$, $j'$)
                \Comment{ Mise à jour couplage local }
                \State $q \gets q \,+ $ get\_interaction\_value($i'$, $j'$)
                \Comment{ Nouvelle valeur }
            \EndFor
            \State $Q \gets Q - p + q$
            \Comment{Mise à jour du couplage}
            \State $E \gets -k Q -h M$
            \Comment{Calcul de l'energie}
        \end{algorithmic}
    \end{minipage}}
    \caption{Procédure de mise à jour de l'énergie en pseudo-code.}
    \label{fig:calc}
\end{figure}


\begin{figure}
    \begin{center}
        \begin{verbatim}
          Children        Self       ...       Symbol
          ...
          +   81.17%     0.41%       ...       update    
          +   75.87%     2.28%       ...       calc_all_one_change
          +   67.41%    22.76%       ...       calc_inter_one_spin
          +   14.56%    14.56%       ...       get_cos
          +   12.19%    12.19%       ...       get_sin
          +   11.88%    11.88%       ...       add_inter
          +    8.29%     8.29%       ...       periodic_conditions
          +    7.77%     7.77%       ...       get_dist                 
          ...
        \end{verbatim}
    \end{center}
    \caption{Resultats de \texttt{perf} sur le surcoût de calcul pour le programme principal. La
    fonction \texttt{update} est décrite en figure \ref{fig:update} et la fonction
    \texttt{calc\_all\_one\_change} est décrite en \ref{fig:calc}. Les fonctions \texttt{get\_*} ne
    servent qu'à lire les valeurs des différents tableaux, \texttt{add\_inter} ajoute une valeur
    donnée sur une partie de la grille des couplages locaux, et \texttt{periodic\_conditions} fait
    la transformation d'indice qui simule les conditions periodiques. }
    \label{fig:perfdata}
\end{figure}

\subsection{Fonctionnement des scripts} 

\subsubsection{Script de lancement du code principal}

Ce script est un ``wrapper'' du code principal. Il simplifie l'appel par le nommage des variables, 
par le passage des valeurs par défaut, etc\footnote{On utilise le package python \texttt{argparse}
pour le passage des arguments, plus simple à gérer que les fonctions POSIX en C.}.

Il permet aussi de lancer plusieurs types d'experiences :
\begin{itemize}
    \item \textbf{Simple sampling.} Le script permet alors de nommer automatiquement les dossiers de
        stockage, lance bien sûr automatiquement les $N_\text{simple}$ experiences et compiles tous
        les fichiers de scalaires en un seul pour une analyse postérieur. Le programme principal est
        appelé avec un nombre d'itérations à faire nul. Il s'initialise donc avec une grille
        aléatoire, calcul son énergie, enregistre la grille ainsi que ses valeurs scalaire et
        quitte.
    \item \textbf{Sweep.} Il s'agit de parcourir un ensemble de paramètres et de lancer une
        simulation pour chaque valeur. Contrairement à ce qui était prévu initialement, on se
        concentre sur des ensembles de couples $(h, k)$ : on peut donc lancer automatiquement 
        $n_h\times n_k$ simulation avec $\{(h_i, k_j)\}_{h_0 < \hdots < h_{n_h}, k_0 < \hdots < k_{n_k}}$.
        En plus de lancer les experiences automatiquement, le script nomme automatiquement les
        dossiers de stockages. 
    \item \textbf{Parallel sweep.} Même chose pour que pour le sweep, mais les intances du code
        principale s'executent en parallèle. Autant de processus qu'il y a de simulations à faire
        sont créés, et on laisse l'OS partager le calcul, ce qui n'est pas optimal mais qui est plus
        rapide à mettre en place. 
\end{itemize}


\subsubsection{Scipts d'analyse des données}

Ces scripts prennent des dossiers de données formattés par les scripts de lancement présentés
précedemment. Pour une seule experience, plusieurs visualisations sont possibles :
\begin{itemize}
    \item Tracé des scalaires $m$, $q$ et $e$ en fonction du nombre d'itération ;
    \item Tracé de $q$ en fonction de $m$ ;
    \item Tracé de grilles $\theta_{ij}$ (une seule, un ensemble, toutes, la dernière), si
        plusieurs grilles sont demandées, elles sont affichées sous forme d'une film à 5 fps par
        défault ;
    \item Tracé de la fonction de d'auto-corrélation radiale $C(r)$.
\end{itemize}

La fonction de d'auto-correlation radiale peut être vue comme la moyenne des correlations 
\begin{equation}
    \begin{aligned}
        C(r) &= \left< \sigma_i \sigma_j\right>_{i^2+j^2=r^2} 
        &= \sum_{\substack{i, j \\ i^2+j^2=r^2}} \sigma_i \sigma_j
    \end{aligned}
\end{equation}
Le processus utilisé pour la calculer est décrit en figure \ref{fig:corr}. Le calcul de cette
fonction est relativement lourd, il est cependant fondamental pour observer la transition de phase.
On s'attend donc à trouver une fonction de la forme 
\begin{equation}
    C_h(r) = b e^{-ar} \iff \log(C_h(r)) = -a r + b
\end{equation}
pour des températures au delà de la transition de phase, et 
\begin{equation}
    C_l(r) = \frac{1}{b+ar} \iff \frac{1}{C_l(r)}  = a r + b  
\end{equation}
pour des températures inférieure à la transition de phase. Comme on peut le
voir, il est facile de transformer cette fonction en équation linéaire, ce qui
peut s'avérer utile pour reconnaitre l'un ou l'autre des régimes. On va donc chercher à
\emph{fitter} ces deux droites affines et noter les $R^2$ pour essayer d'observer une transition de
comportement.


\begin{figure}
    \fbox{
    \begin{minipage}[]{\textwidth}
        \begin{algorithmic}
            \Require prod\_shift(lattice, i, j) qui retourne un tableau tel que $A_{kl} =
            B_{kl}B_{(k+i)(j+l)}$ avec des conditions periodiques
            \Require sum(lattice) qui ajoute toutes les valeurs d'une grille 
            \Require abs\_sort($X, Y$) tri les valeurs des listes $X$ et $Y$ rapport à $X$
            \State $X \gets \emptyset,~ Y \gets \emptyset$
            \For { $i, j$ in $[0, N/4]\times[0, N/4]$ }
            \Comment{Boucle sur le voisinage à $N/4$ dans une direction}
                \State push\_item($X$, $\sqrt{i^2+j^2}$)
                \Comment{Calcul du rayon}
                \State $ S^{(++)} \, = $ prod\_shift(lattice,$ i, j$)
                \Comment{Correlations par direction}
                \State $ S^{(+-)} \, = $ prod\_shift(lattice,$ i, -j$)
                \State $ S^{(-+)} \, = $ prod\_shift(lattice,$ -i, j$)
                \State $ S^{(--)} \, = $ prod\_shift(lattice,$ -i, -j$)
                \State push\_item$\left( Y, \text{sum}\left( \frac{S^{(++)} + S^{(+-)} + S^{(-+)} +
                S^{(--)}}{4 N^2} \right)\right)$
                \Comment{Moyenne des correlations directionnelles}
            \EndFor
            \State $X, Y \gets $ abs\_sort($X,Y$)
        \end{algorithmic}
    \end{minipage}}
    \caption{Procédure de calcul de la fonction d'autocorrelation.}
    \label{fig:corr}
\end{figure}

\bibliographystyle{alpha}
\bibliography{rapport}

\end{document}

